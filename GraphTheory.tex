\chapter{Graph Theory and Computational Social Science}\label{ch:GraphTheory}

Graph theory is the study of mathematical structures, called graphs, which are
used to model pairwise relations between entities. Graphs consist of a finite
set of vertices, $V$, and a set of ordered pairs of vertices, $E$, called edges.
A graph can be defined by the tuple, $G=(V,E)$. The graphs built in this project
have added constraints and is defined as below:

\begin{itemize}
    \item \emph{Vertices}:% Let $V_{1}=\{v_{1},{v_{2},...,{v_{n}\}$
    \item The text in the entries may be of any length.
  \end{itemize}


\subsection{Spectral Graph Theory}\label{sec:spectralGraphTheory}

    An excerpt from an Alloy
    
\subsubsection{Network Laplacian Spectral Descriptor}\label{sec:NetLSD}

 
\section{Random Graphs}\label{sec:RandomGraphs}

    Where the graphs at...
    
\subsection{Background}\label{sec:RandomGraphBackground}

	
\subsection{Stochastic Block Models}\label{sec:SBM}

\begin{quote}
    Stochastic block models
    Statistical methods for block modeling have been proposed as stochastic
    block models. Here the positions are taken to be latent classes for the nodes,
    referred to as the colors of the nodes. The conditional distribution of dyads is
    assumed to be independent conditional on the node colorings and the goal is to
    determine the latent classes, and thereby to define the block model. This
    approach was first proposed by Holland, Laskey, and Leinhardt (1983) and further
    developed by Nowicki and Snijders (2001) who presented an algorithm for node
    classification. Airoldi, Blei, Feinberg, and Xing (2008) extended this approach
    to mixed membership models. \cite{robins2013tutorial}
\end{quote}
    
	
\section{Measures of Centrality}

Centrality is a measure of prominence for vertices within a graph. For the
purpose of this thesis, it will be used to measure the relative importance of
different topics tweeted about in the lead up to Canada's 2019 federal election.

    
\subsection{Background}\label{sec:CentralityBackground}

There are various different ways of measuring vertex centrality that have
successfully been applied to problems in marketing, economics and epidemiology;
Stephenson and Zelen explored the utility of centrality measures in studying the
social dynamics of Gelada baboons. \cite{stephenson1989rethinking}. Common
centrality measures include measures of degree and betweenness. This thesis will
focus on the notion that central vertices are close to other central vertices,
which is one of the founding intuitions behind Google’s “page-rank” algorithm
and eigenvector centrality. 

	
\subsection{Eigenvector Centrality}\label{sec:EigCentrality}

As Newman lays out in his 2016, Mathematics of Networks: ``the eigenvector
centrality [...] accords each vertex a centrality that depends both on the
number and the quality of its connections: having a large number of connections
still counts for something, but a vertex with a smaller number of high-quality
contacts may outrank one with a larger number of mediocre contacts.'' \cite{newman2008mathematics}

If we define the eigencentrality of vertex $x$ as $C_{E}(x)$, where $C_{E}(x)$
is proportional to the average eigenvector centrality of $x$'s neighbours,
multiplied by some constant $\lambda$:
\begin{equation}
    C_{E}(x)=\frac{1}{\lambda}\sum_{j=1}^{n}A_{xj}C_{E}(x)
\end{equation}
By defining the vector of centralities as $C_E(X) = (C_E(x_1),C_E(x_2),...)$
this equation can be rewritten as $\lambda C_E(X) = \lambda A$, and it is
evident that CE(V) is an eigenvector of the adjacency matrix with eigenvalue, $\lambda$
\cite{newman2008mathematics}. By Perron-Frobenius theorem, picking the largest eigenvalue of $A$ will
result in all elements of $C_E(X)$ being non-negative \cite{newman2008mathematics}.