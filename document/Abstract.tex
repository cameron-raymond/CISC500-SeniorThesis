The advent of social media has enabled political parties to engage with the
broader populous in new and unforeseen ways. This, coupled with rising levels of
political polarization has prompted debates as to whether people care about
policy anymore, or if they self-select into political bubbles online based on
their chosen party leader. This thesis proposes a novel adaptation of stochastic
blockmodelling to measure the degree to which political engagement on Twitter is
driven by policy or party leaders. Building on a graph theoretical approach,
measures of topic centrality are developed to give a metric for how efficient
topics were at either rallying or spanning party leaders' bases. This is done in
the context of the 2019 Canadian federal election.
