\chapter{Conclusion}\label{ch:Conclusion}

\section{Summary}

This thesis has theoretical and empirical contributions. First, measures of
topic centrality (total network topic centrality, and party leader topic
centrality) were introduced, which allows for the investigation of the relative
centrality of policies to the broader populous versus individual party leaders.
Second, a novel adaptation of stochastic blockmodelling was developed that
allows different axes of political engagement to be compared and contrasted.
This was extended to include a deep stochastic blockmodel, that used artificial
neural networks to calculated edge probabilities. 

As for the empirical results: through the topic centrality measures, it became
clear that Maxime Bernier's rhetoric surrounding immigration, free speech and
his claims of a ``cult of diversity'' did not gain traction within the broader
populous -- and had a disproportionately low impact on his base relative to the
number of tweets pertaining to those subjects. Additionally, while tweets
pertaining to the SNC Lavalin affair, and carbon taxes had high engagement rates
in the total network, they generally had average engagement rates for individual
party leaders' bases. 

Through fitting the stochastic hybrid blockmodel (SHBm) to the original
engagement graph, it became clear that both policy and party leaders were
important in driving engagement in the run up to Canada's 2019 federal election.
The standard SHBm showed a clear parabolic trend -- where models that privileged
party leaders over topics, or topics over party leaders -- performed worse than
models that took both into account. While without a longitudinal analysis it is
hard to make a definitive call on Ezra Klein's argument -- that there has been a
shift towards polarization along the axis of party rather than policy -- it does
call into question popular notions of what politically engaged individuals care
about. 

\section{Other Work}

As dictated by the Queen’s School of Computing, 25\% of the evaluation for CISC
500 is left to be arranged between the student and their supervisor. Given the
nature of this research, lab work or developing software appeared out of scope.
I believe that one of the benefits of applied research is its ability to take
abstract concepts and frameworks and show their utility in solving complicated
problems. This took the form of performing a lecture on the project's research
and the application of graphs in Dr. Robin Dawes’ CISC 235: Data Structures
course.

Additionally, the findings of this thesis were originally going to be presented
at the International Network for Social Network Analysis' annual Sunbelt
Conference. Unfortunately, due to the ongoing COVID-19 pandemic, this engagement
was cancelled.

\section{Future Work}

Many of the contributions of this thesis are novel and have ample room for
refinement. Most notably, the deep SHBm in section \ref{sec:DeepSBMs}, produced
unexpected results. Primarily, this is a result of the simplicity of the neural
networks used and more realistic results could likely be derived by introducing
recurrent neural networks like LSTMs instead. Additionally, the actual algorithm
for generating the stochastic blockmodels (algorithm \ref{algorithm:SBM}), could
be refined further to create more lifelike representations.

In terms of extending the work that has already been done: longitudinal analysis
on what policies party leaders tweet about, what topics drive the most
engagement, and along what axes users engage with elites online is likely to
derive interesting results. Additionally, latitudinal analysis comparing
different axes of engagement in states like the United Kingdom, France, Chile
and the US would help situate Canada's unique political system in the global
sphere. 