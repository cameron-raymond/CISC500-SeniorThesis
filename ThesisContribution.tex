\chapter{Thesis Contribution}\label{ch:ThesisCont}
      
\section{Topic Modelling}\label{sec:TopicModelling}

In order to evaluate the relative importance of policy and party leaders in
driving political engagement on Twitter, all the tweets collected must first be
organized by topic. In order to do so, a latent Dirichlet allocation (LDA) was
trained on the English tweets of Canada's five major, english speaking party
leaders: Andrew Scheer, Elizabeth May, Jagmeet Singh, Justin Trudeau, and Maxime
Bernier. The timeframe of collection ranges from October 21, 2018 to October 21,
2019 - the eve of Canada's federal election. While the tweets from each Federal
party's official accounts were also collected, they predominantly acted as
logistical tools -- informing party affiliates of events and rallies. The
personal accounts for party leaders were generally more pertinent to their
beliefs, platforms and style of rhetoric, and thus are better suited in this
context. In this spirit, only tweets of the party leader were collected,
excluding retweets. Figure \ref{fig:tweets_over_time} visualizes the daily and
cumulative number of tweets over time, in aggregate and by party leader,
resulting in 7978 total tweets.

\begin{singlespacing}
    \begin{figure}[H]
    \centering
    \includegraphics[scale=0.40]{Figures/tweets_over_time}
    \caption[Daily and Cumulative Tweets over Time]{Daily and Cumulative Tweets over Time}
    \label{fig:tweets_over_time}
    \end{figure}
\end{singlespacing}

\subsubsection{Text Cleaning}

Given the inherent noise and extraneous info in text data, it is standard and
necessary to preprocess text before modelling \cite{sapul2017trending}. The text
cleaning pipeline removes punctuation marks, stop words, words with fewer than
three characters, and URLS, as well as common twitter symbols like ``RT:'',
``@'' and ``\#''. Emojis were converted to text using the python package
\texttt{emoji}. After this process, all text was converted to lower-case and
lemmatized to get rid of common suffixes. Therefore the tweet in figure
\ref{fig:tweet_ex} after preprocessing reads: \emph{wherever maple leaf fly
represents rich history bright future value hold dear happy flag day canada}.

\begin{singlespacing}
    \begin{figure}[H]
    \centering
    \includegraphics[scale=0.55]{Figures/tweet_ex}
    \caption[Example Tweet]{Example Tweet}
    \label{fig:tweet_ex}
    \end{figure}
\end{singlespacing}

\subsection{Hyper-Parameter Tuning}\label{sec:TopicModellingHP}

As discussed in section \ref{ch:TopicModelling}, the LDA takes in three
parameters: $\alpha$ - which acts as a concentration parameter for how documents
are modelled as topics; $\beta$ - which acts as a concentration parameter for
how topics are modelled as words; and $k$ which is the number of topics to be
modelled. By performing a parameter sweep, where $\alpha$ and $\beta$ lie on the
interval $\left[0,1\right]$ with increments of 0.05, and $k$ ranges between 4
and 7, the LDA was exposed to the entire corpus and then evaluated using c\_v
coherence. Figure \ref{fig:lda_param_sweep} shows, for each $k$ value, the c\_v
coherence as a function of different combinations of $\alpha$ and $\beta$.

\begin{singlespacing}
\begin{figure}
    \centering
    \begin{tabular}{cc}
      \includegraphics[width=65mm]{Figures/Coherence_Surface_k=4} &
      \includegraphics[width=65mm]{Figures/Coherence_Surface_k=5} \\
    (a) $k=4$ & (b) $k=5$ \\[6pt]
     \includegraphics[width=65mm]{Figures/Coherence_Surface_k=6} &
     \includegraphics[width=65mm]{Figures/Coherence_Surface_k=7} \\
    (c) $k=6$ & (d) $k=7$ \\[6pt]
    \end{tabular}
    \caption[LDA Parameter Sweep Results]{LDA Parameter Sweep Results}
    \label{fig:lda_param_sweep}
\end{figure}
\end{singlespacing}


\subsection{Results}\label{sec:TopicModellingRes}

After performing the parameter sweep described in section
\ref{sec:TopicModellingHP}, the most performant model had a $k$ value of 7,
$\alpha$ of 0.31 and $\beta$ of 0.81 and a c\_v coherence score of 0.48. By
labelling each tweet as the maximum probability value in its topic mixture, each
tweet was assigned a single topic. The word clouds for each topic are described
in figure \ref{fig:topic_word_clouds}. 

\begin{singlespacing}
    \begin{figure}
        \centering
        \begin{tabular}{ccc}
        \includegraphics[width=45mm]{Figures/topic_1_wordcloud} &
        \includegraphics[width=45mm]{Figures/topic_2_wordcloud} &
        \includegraphics[width=45mm]{Figures/topic_3_wordcloud} \\
        (a) Topic 1 & (b) Topic 2 & (c) Topic 3  \\[6pt]
        \includegraphics[width=45mm]{Figures/topic_4_wordcloud} &
        \includegraphics[width=45mm]{Figures/topic_5_wordcloud} &
        \includegraphics[width=45mm]{Figures/topic_6_wordcloud} \\
        (d) Topic 4 & (e) Topic 5 & (f) Topic 6  \\[6pt]
        \multicolumn{3}{c}{\includegraphics[width=45mm]{Figures/topic_7_wordcloud}
        }\\
        \multicolumn{3}{c}{(g) Topic 7}
        \end{tabular}
        \caption[LDA Topic Word Clouds]{LDA Topic Word Clouds}
        \label{fig:topic_word_clouds}
    \end{figure}
\end{singlespacing}

Topic 1 pertained to campaign messages, rallies and logistics -- and makes up
8.2\% of all tweets. Topic 2 contains tweets regarding a carbon tax, pipelines
and the economy -- and makes up 16.3\% of all tweets. Topic 3 contains tweets
about the SNC Lavalin affair, a scandal that plagued Justin Trudeau, and tweets
about corruption -- making up 18\% of all tweets. Topic 4 is predominantly
tweets appealing to the middle-class and economy -- and is 29.7\% of all tweets;
topic 5 contains celebratory messages about the campaign, as well as tweets
regarding national holidays and days of remembrance -- and make up 15\% of all
tweets. Topic 6 is made up of tweest about immigration, diversity and free
speech -- and makes up 11.5\% of all tweets. Finally, topic 7 contains
tweets regarding healthcare, abortion and pharmacare -- and makes up 1\% of all
tweets. The magnitude of how many tweets were assigned to each topic is shown in
figure \ref{fig:topic_distribution}. The vertices representing tweets of different
topics in figure \ref{fig:og_graph} are assigned different colours. 

\begin{singlespacing}
    \begin{figure}[H]
    \centering
    \includegraphics[scale=0.2]{Figures/topic_distribution}
    \caption[LDA Topic Distribution]{LDA Topic Distribution}
    \label{fig:topic_distribution}
    \end{figure}
\end{singlespacing}

\section{Topic Centrality}\label{sec:TopicCentrality}

\subsection{Total Network Topic Centrality}\label{sec:NetTopicCentrality}

	Define what it is, show the full graph again.

\subsection{Party Leader Topic Centrality}\label{sec:LeaderCentrality}

    Now, consider a fact stating that relation \texttt{r} is total, i.e.,

\subsection{Results}\label{sec:TopicCentralityResults}


\section{Stochastic Block Models}\label{sec:SBMs}

\begin{quote}
Stochastic block models
Statistical methods for block modeling have been proposed as stochastic
block models. Here the positions are taken to be latent classes for the nodes,
referred to as the colors of the nodes. The conditional distribution of dyads is
assumed to be independent conditional on the node colorings and the goal is to
determine the latent classes, and thereby to define the block model. This
approach was first proposed by Holland, Laskey, and Leinhardt (1983) and further
developed by Nowicki and Snijders (2001) who presented an algorithm for node
classification. Airoldi, Blei, Feinberg, and Xing (2008) extended this approach
to mixed membership models. \cite{robins2013tutorial}
\end{quote}

Describe generic algorithm without specifying if its the party/topic model

\subsection{Artificial Neural Network Adaptation}\label{sec:ANNAdaptation}

\subsubsection{Stochastic Party Leader Block Model}\label{sec:SPLBM}

The stochastic party leader model generates user behaviour only taking into
account the previous party leaders each user had engaged with prior. For each
user, when deciding which tweet they are to retweet, their retweet history
(ex.\ $history=[JT=0,AS=1,JS=3,EM=2,MB=0]$) is converted into a probability
distribution (ex.\ $probs=[0.09,0.18,JS=0.36,EM=0.27,MB=0.09]$). In this sense,
it models a world in which politically engaged Twitter users only engage along
the axis of party leaders, with a complete disregard for the topics tweeted
about.

\subsubsection{Stochastic Topic Block Model}\label{sec:STBM}

Conversely, the stochastic topic block model models a world in which politically
engaged Twitter users only engage along the axis of topics. Here, the topic
history of a user is converted into a probability distribution, and with a
probability of $\epsilon$, that user will retweet the topic with the highest
activation - regardless of which party leader tweeted it. 

\subsubsection{Stochastic Hybrid Block Model}\label{sec:SHBM}

The final model developed is a hybrid of the stochastic party leader block
model, and the stochastic topic block model. Here, two history vectors for each
user are captured - the $n$ dimensional party leader history vector, and the
$k$ dimensional topic history vector. After each respective vector is converted
into a probability distribution, the weight of a retweet of
topic $i$ by party leader $j$ is determined by the function:

\begin{equation}
    weight(party leader=i, topic=j)=\alpha P(i)+(1-\alpha)P(j)
\end{equation}

Where $P(i)$ is index $i$ of that user's \emph{party leader} probability
distribution, $P(j)$ is index $j$ of that user's \emph{topic} probability
distribution, and $\alpha$ is some constant that determines the relative
weighting of the two. As $\alpha$ approaches $1$, the hybrid model becomes
equivalent to the stochastic party leader block model - and as $\alpha$
approaches $0$ the model approaches the stochastic topic block model. This model
then generates different "worlds" in which users' political engagement falls on
the spectrum from only caring about \emph{party leaders} to only caring about
\emph{topics}.

\subsection{NetLSD for Describing Political Engagement}\label{sec:NetLSDForSBM}

The final objective of comparing the relative importance of topics and party
leaders in driving political engagement requires comparing the structure of
target graph shown in \ref{sec:graphTheoryBackground}, and various hybrid models
generated with different values of $\alpha$. Using the Network Laplacian
Spectral Descriptor described in section \ref{sec:NetLSD} by Tsitsulin et al.
the optimal alpha value can be determined in a scale-adaptive, size-invariant,
and permutation-invariant manner \cite{netlsd}.

\subsection{Results}\label{sec:SBMsResults}



