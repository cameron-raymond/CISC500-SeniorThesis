\chapter{Introduction}

\section{Background}

The advent of social media has enabled political parties to engage with the
broader populous in new and unforeseen ways. The ability to bypass the
traditional mediating forces of mass media allows for an unfiltered promotion of
policy, ideology and party stances. This is specifically interesting in Canada’s
political system which has historically been defined by large brokerage parties.
In order to win a diverse range of electoral districts across Canada, these “big
tent” parties try to appeal to various political persuasions. Political adverts,
and policy have traditionally been the conduits through which brokerage parties
attempt to accommodate different ideologies, but social media allows for a
direct, granular approach to political messaging which is completely novel.
Social networks, formed through new media like Twitter, are inherently
relational and thus lend themselves well to being represented as graphs.
Therefore, as political strategy becomes increasingly digital, the use of graph
theory can potentially illustrate how large brokerage parties organize and along
what axes Canadians engage with political parties.

\section{Motivation}\label{sec:motivation}

The way information is distributed and received has changed significantly over
the past decade. Cogburn and Espinoza-Vasquez argue that Barrack Obama’s 2008
presidential campaign was a watershed moment in social media campaigning -- and
in the subsequent decade, from Macron to Brexit to the Five Star Movement,
social media has played an increasing role in how politics is conducted
\cite{cogburn2011networked}. Between 2013 and 2018, the share of Canadian
federal media expenditure spent on digital advertising rose from 27\% to 65\%, a
140\% increase, making the study of new media critical from a social science
perspective \cite{annualReportCanadaAdvertisingActivities_2018}. Additionally,
rises in political polarization, populism and a decline in trust in political
institutions in the $21^{st}$ century has been a topic of popular debate. Ezra
Klein argues in his 2019 book, \emph{Why We're Polarized}, that this due to a
shift in preferences for parties over policies \cite{levitsky2018democracies}.
If this is the case, then this preference to engage along party lines rather
than choosing to engage with specific issues should pattern engagement. An
empirical analysis of how users behave and engage with political parties online
should privilege the relational aspect of social media. Social network analysis
helps avoid the pitfalls of survey data, famously described by Allen Barton as
``a sociological meat grinder, tearing the individual from [their] social
context'' \cite{freeman2004development}.

Graph theory’s use in social network analysis, also called network science, has
already been applied to explore problems in marketing, sociology and
epidemiology -- but there is a gap in analysis of political engagement online.
Therefore, the contribution of this thesis is a novel, robust mathematical
process for analyzing different axes of political engagement online in a purely
relational manner. A secondary outcome of making relationships a “first-class
citizen” in this work will be an organic analysis of what issues produce the
most engagement. This deviates from traditional survey data that ask test
subjects which issues concern them -- and instead uses observations of past
behaviour to model variations in local connectivity to answer the question:“what
do people actually care about?”All of this will be done in the context of the
2019 Canadian federal election and the tweets of Canada's five major, english
speaking party leaders: Andrew Scheer, Elizabeth May, Jagmeet Singh, Justin
Trudeau, and Maxime Bernier. 

\section{Research Question}

The primary question concerning this project is: in the lead up to the 2019
Canadian federal election, did politically active users on Twitter engage with
political elites along the axis of issues\footnote{The terms policy, issue and topic will be used interchangeably to
refer to categories of messages.} or parties? If Ezra Klein is correct, then \emph{who} produces the
message will pattern engagement more than \emph{what} the message is; this will
act as the initial null hypothesis, with the alternate hypothesis being that who
produces the message is equally to or less important than what the message is. 

The secondary question to be explored is: during this period, what topics
produced the highest level of engagement? Also, what topics spanned
multiple party leaders' bases, indicating a bridging of different ideologies,
and what topics rallied party leaders' bases?