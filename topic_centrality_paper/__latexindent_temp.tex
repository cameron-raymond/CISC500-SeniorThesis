% This is file NWSguide.tex
% release v1.00, 12th June 2012
%   (based on JFPguide.tex v1.11 for LaTeX 2.09)
% Copyright (C) 2012 Cambridge University Press

\NeedsTeXFormat{LaTeX2e}

\documentclass{nws}

%%% Macros for the guide only %%%
\providecommand\AMSLaTeX{AMS\,\LaTeX}
\newcommand\eg{\emph{e.g.}\ }
\newcommand\etc{\emph{etc.}}
\newcommand\bcmdtab{\noindent\bgroup\tabcolsep=0pt%
  \begin{tabular}{@{}p{10pc}@{}p{20pc}@{}}}
\newcommand\ecmdtab{\end{tabular}\egroup}
\newcommand\rch[1]{$\longrightarrow\rlap{$#1$}$\hspace{1em}}
\newcommand\lra{\ensuremath{\quad\longrightarrow\quad}}

\title[Measures of Topic Centrality for Online Political Engagement]
      {Measures of Topic Centrality for Online Political Engagement}

 \author[C.J.K Raymond]
        {Cameron Raymond\\
         Queen's University School of Computing, Kingston K7L 3N6, CA\\
         \email{c.raymond@queensu.ca}}

\jdate{May 2020}
\pubyear{2020}
\pagerange{\pageref{firstpage}--\pageref{lastpage}}
% \doi{S0956796801004857}

% \newtheorem{lemma}{Lemma}[section]

\begin{document}

\label{firstpage}

\maketitle

\begin{abstract}
  The advent of social media has enabled political parties to engage with the
  broader populous in new and unforeseen ways -- and the ability to bypass the
  traditional mediating forces of mass media allows for an unfiltered promotion
  of policy, ideology and party stances. Social networks, formed via social
  media like Twitter, are inherently relational and thus lend themselves well to
  being represented as graphs. The uses of graphs can help measure how political
  elites promote different categories of messages, and how the electorate engage
  along various axes. This article proposes two novel measures of topic
  centrality, which measure how central messages of various topics were to a party
  leader's core voting group, or to the broader discourse. Statistically
  significant variations in topic centrality are shown in the 2019 Canadian
  Federal Election.
\end{abstract}

\tableofcontents

\section{Introduction}

The way information is distributed and received has changed significantly over
the past decade. As Cogburn and Espinoza-Vasquez argue, Barrack Obama’s 2008
presidential campaign was a watershed moment in social media campaigning – and
in the subsequent decade, from Macron to Brexit to the Five Star Movement,
social media has played an increasing role in how politics is conducted \cite{cogburn2011networked}. The
same holds true for Canada, between 2013 and 2018 the share of Canadian federal
media expenditure spent on digital advertising rose from 27\% to 65\%, a 140\%
increase, making the study of new media critical from a social science
perspective \cite{annualReportCanadaAdvertisingActivities_2018}.

\subsection{Social Media and Political Communication}

While it is clear that technology is changing how information is received, and
thus also changing how politics is conducted, it may not be clear the role of
Canadian politics in this context. However, Canada’s political system is a
fertile environment to test the importance of political messaging, because
relative to most liberal democracies, the system is dominated by party
politicians. As Carty put it: 
\begin{quote}
  No obvious simple geographic reality, no common
linguistic or religious homogeneity, no common revolutionary experience or
unique historical moment animated [Canada] or gave it life. Canada was created
when a coalition of party politicians deemed it to be in their interest to do
so, and it has been continuously grown, reshaped and defended by its
politicians.
\end{quote}
 [4] Thus, it is not surprising that Canada’s electoral system
encourages electoral pragmatism – and developed large, “big tent” parties that
are among the most or- ganizationally weak and decentralized of established
democracies [4]. This system defines political parties as brokers of the often
conflicting, weakly integrated elec- torate – as opposed to mobilizers of
distinct communities, articulating claims rooted in their pre-existing
interests. In this way, parties act as the principal instruments of national
accommodation, rather than democratic division [4].


\subsection{Eigenvector Centrality}

\section{Methods}

\subsection{Data}

\subsection{Topic Modeling}

\subsection{Topic Centrality}

\subsubsection{Total Network Topic Centrality}

\subsubsection{Party Leader Topic Centrality}

\section{Results}

\subsection{Topic Saliency}

\subsection{Total Network Topic Centrality}

\subsection{Party Leader Topic Centrality}

\section{Discussion}

\bibliographystyle{plain}
\bibliography{bibliography}

\label{lastpage}

\end{document}

% end of NWSegui.tex