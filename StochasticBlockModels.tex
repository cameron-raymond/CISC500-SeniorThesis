\section{Stochastic Block Models}\label{sec:SBMs}

Describe generic algorithm without specifying if its the party/topic model

\subsection{Artificial Neural Network Adaptation}\label{sec:ANNAdaptation}

\subsubsection{Stochastic Party Leader Block Model}\label{sec:SPLBM}

The stochastic party leader model generates user behaviour only taking into
account the previous party leaders each user had engaged with prior. For each
user, when deciding which tweet they are to retweet, their retweet history is
converted into a probability distribution (ex.\
$history=[JT=0,AS=1,JS=3,EM=2,MB=0]$ generates
$probs=[0.09,0.19,JS=0.36,EM=0.27,MB=0.09]$). In this sense, it models a world
in which politically engaged Twitter users only engage along the axis of party
leaders, with a complete disregard for the topics tweeted about.

\subsubsection{Stochastic Topic Block Model}\label{sec:STBM}

Conversely, the stochastic topic block model models a world in which politically
engaged Twitter users only engage along the axis of topics. Here, the topic
history of a user is converted into a probability distribution, and with a
probability of $\epsilon$, that user will retweet the topic with the highest
activation -- regardless of which party leader tweeted it. 

\subsubsection{Stochastic Hybrid Block Model}\label{sec:SHBM}

The final model developed is a hybrid of the stochastic party leader block
model, and the stochastic topic block model. Here, two history vectors for each
user are captured -- the $n$ dimensional party leader history vector, and the
$k$ dimensional topic history vector. After each respective vector is converted
into a probability distribution, the weight of a retweet of
topic $i$ by party leader $j$ is determined by the function:

\begin{equation}
    weight(party leader=i, topic=j)=\alpha P(i)+(1-\alpha)P(j)
\end{equation}

Where $P(i)$ is index $i$ of that user's \emph{party leader} probability
distribution, $P(j)$ is index $j$ of that user's \emph{topic} probability
distribution, and $\alpha$ is some constant that determines the relative
weighting of the two. As $\alpha$ approaches $1$, the hybrid model becomes
equivalent to the stochastic party leader block model -- and as $\alpha$
approaches $0$ the model approaches the stochastic topic block model. This model
then generates different ``worlds'' in which users' political engagement falls on
the spectrum from only caring about \emph{party leaders} to only caring about
\emph{topics}.

\subsection{NetLSD for Describing Political Engagement}\label{sec:NetLSDForSBM}

The final objective of comparing the relative importance of topics and party
leaders in driving political engagement requires comparing the structure of
target graph shown in \ref{sec:graphTheoryBackground}, and various hybrid models
generated with different values of $\alpha$. Using the Network Laplacian
Spectral Descriptor described in section \ref{sec:NetLSD} by Tsitsulin et al.
the optimal alpha value can be determined in a scale-adaptive, size-invariant,
and permutation-invariant manner \cite{netlsd}.

\subsection{Results}\label{sec:SBMsResults}

