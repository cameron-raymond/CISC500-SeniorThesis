\chapter{Social Media and Politics}\label{ch:SMandPolitics}

\subsection{Political Communication in the Digital Era}

Technological innovation often precedes political disruption; Martin Luther’s 95
theses could only be spread at scale by virtue of Gutenberg inventing his
printing press first \cite{gardels2019renovating}. This example had obvious
political ramifications and is a testament to the impact of the means of
communication on the political sphere. McNair defines political communication
as:

\begin{enumerate}
    \item All forms of communication undertaken by politicians and other
    political actors for the purpose of achieving specific objectives.
    \item Communication addressed to these actors by non-politicians such as
    voters and newspaper columnists.
    \item Communication about these actors and their activities, as contained in news reports, editorials, and other forms of media discussion of politics. \cite{mcnair2017introduction}
  \end{enumerate} 
This is a broad, outward focused definition that includes most public, political
discourse - verbal or otherwise. However, social media has changed all three
aspects of this definition in novel ways. Most notably, the rate at which
politicians can communicate information to achieve “specific objectives” has
increased as a result of rapid nature of social media, and the granularity at
which actors can target the these messages has gotten smaller due to the massive
swaths of user data available \cite{nickerson2014political}. Therefore, the
value of using computational methods to study political communication in the
$21^{st}$ century is a natural fit.

\subsubsection{Technologies Implications for Democratization}


\subsection{Canadian Brokerage Politics}

ahh yeah boiii. Explain what brokerage parties are

\subsubsection{Rationale}

Why canada is a good case for this kind of question