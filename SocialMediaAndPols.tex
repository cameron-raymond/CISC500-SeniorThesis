\chapter{Social Media and Politics}\label{ch:SMandPolitics}

\section{Political Communication in the Digital Era}

Technological innovation often precedes political disruption; Martin Luther’s 95
theses could only be spread at scale by virtue of Gutenberg inventing his
printing press first \cite{gardels2019renovating}. This example had obvious
political ramifications and is a testament to the impact of the means of
communication on the political sphere. McNair defines political communication
as:

\begin{enumerate}    
    \item All forms of communication undertaken by politicians and other
    political actors for the purpose of achieving specific objectives.
    \item Communication addressed to these actors by non-politicians such as
    voters and newspaper columnists.
    \item Communication about these actors and their activities, as contained in news reports, editorials, and other forms of media discussion of politics. \cite{mcnair2017introduction}
\end{enumerate} 

This is a broad, outward focused definition that includes most public, political
discourse -- verbal or otherwise, and social media has changed all three aspects
of in novel ways. Most notably, the rate at which politicians can communicate
information has increased as a result of rapid nature of social media, and the
granularity with which actors can target the these messages has gotten smaller
due to the massive swaths of user data available \cite{nickerson2014political}.
The increasing shift towards using social media for the purposes of political
communication, and the ease with which these messages and the users who engaged
with them can be collected and analyzed demonstrate the value of using
computational methods to study political communication in the $21^{st}$ century.

\subsection{Technology's Implications for Democratization}

While fully exploring the impact of social media on democracy is out of the
scope of this project, it is an important justification for why research in this
area needs to be conducted. The traditional model of the media being a mediating
force through their reporting, commentary and analysis is no longer valid.
Previously, political actors needed to use media like television, radio, and
newspapers to broadcast their messages to their desired audiences
\cite{mcnair2017introduction}. Social networking sites allow these same actors
to reach audiences in the millions without having to gain access to the media
first; in this effect, the second and third elements of McNair’s definition are
also being transformed. Yascha Mounk argues that this has given voice to
political outsiders who would be shut out from mainstream platforms. Thus,
social media may not be inherently democratic or undemocratic, as it has
contributed to democratic backsliding and overturning authoritarian governments,
but it can certainly have a destabilizing affect \cite{mounk2018people}.
Therefore, a better understanding of new media’s ramifications is critical, and
empirical modeling can aide in this understanding.

\section{Canadian Brokerage Politics}

While it is clear that technology is changing how information is received, and
thus also changing how politics is conducted, it may not be clear the role of
Canadian politics in this context. However, Canada’s political system is a
fertile environment to test the importance of political messaging, because
relative to most liberal democracies, the system is dominated by party
politicians. As Carty put it:

\begin{quote}
No obvious simple geographic reality, no common linguistic or religious
homogeneity, no common revolutionary experience or unique historical moment
animated [Canada] or gave it life. Canada was created when a coalition of party
politicians deemed it to be in their interest to do so, and it has been
continuously grown, reshaped and defended by its politicians.
\cite{carty2010political}
\end{quote}

Thus, it is not surprising that Canada’s electoral system encourages electoral
pragmatism -- and developed large, “big tent” parties that are among the most
organizationally weak and decentralized of established democracies
\cite{carty2010political}. This system defines political parties as brokers of
the often conflicting, weakly integrated electorate -- as opposed to mobilizers
of distinct communities, articulating claims rooted in their pre-existing
interests. In this way, parties act as the principal instruments of national
accommodation, rather than democratic division \cite{carty2010political}.

\subsubsection{Rationale}

The dominance of parties in Canadian politics, their amorphous ideological
stances, and the many intersectional geographic, linguistic and religious
cleavages have given birth to what’s been coined the brokerage party system.
\cite{carty2010political}. The need to capture pluralities in a diverse range of
electoral districts means that most parties have to take stances on most issues,
and thus when a user engages with a specific issue, it doesn't necessarily
invoke a specific party or vice versa. This lack of congruence between parties
and issues allows for a more full exploration of the two axes of engagements
described (policy and party) -- giving Canada a unique set of dynamics and
making it an interesting case to explore.